\chapter{Introduction}\label{chap:intro}

The cancer is a disease whose cure is still a subject of several research projects. The cancer has developed several therapies such as chemotherapy and radiotherapy. A new branch of studies aims to develop targeted drugs capable of affect only damaged cells without harming healthy ones still inside the body. This approach involves the production of large amounts of data, making essential the presence of an efficient information management.

Many efforts have been devoted to the development of data management systems related to cancer research. One of the main ICT technologies used in research laboratories is the Laboratory Information Management Systems (LIMS). Many commercial LIMS are available, but are usually very expensive and require a large amount of human and economic resources to fit the specific requirements of each laboratory. However, also exist a number of LIMS proposed by the research community, released as open-source projects. Certain manage data related to some specific research procedures. Other LIMS focus on the management of data associated with the mice used in research, monitoring the various colonies of animals. However, all these systems are mainly focused on the management of some types of highly specific data relating to a particular series of experiments.

To overcome the main problems of the approaches mentioned above and to satisfy the requirements deriving from the activities of research laboratories, the research team DBDMG the Polytechnic of Turin, in collaboration with the staff of IRCC of Candiolo, proposes the Laboratory Assistant Suite (LAS), a platform which assists researchers in the various laboratory activities. Its modular architecture makes it possible to manage different types of raw data (for example, organic molecules), and tracking of experimental data. In addition, the platform supports the integration of different resources when performing a series of tests in order to extract information about the analyzed tumors. The user interfaces are designed to help manage the data in harsh environments, where researchers need to minimize their interactions with the system (for example, under sterile conditions).

At IRCC of Candiolo, researchers perform studies and research on so-called xenopatients. These are immunosuppressed mice in which are inserted fragments of cancer tissue from human subjects, to be able to study the reactions to certain drugs. In this way, the aim is to obtain a specialized drug treatment for each type of tumor under study. Within the LAS \Xeno\ (XMM) is the operational level. is a module that can send information requests to the modules above. Its task is to manage data and experiments on xenopatients. 

XMM aims to track and monitor all activities related to xenopatients and their life cycle, starting from the moment they are delivered from suppliers at the institute. It should then be able to handle these operations, implementing features and interfaces in ways that meet the specific demands by IRCC of Candiolo. An example is the requirement of compatibility with devices for reading barcodes and RFID microchip. In addition, it must be usable by mobile users, such as tablets.

The developed system, named \Xeno\ (XMM), allows you to track and monitor all activities related to research and study xenopatients and their life cycle. The user interfaces are designed to help manage the data in harsh environments, where researchers need to minimize their interactions with the system. XMM offers a work environment capable of simulating the laboratory, where researchers work every day. This allows for reducing the errors during data entry, especially in dedicated interfaces to the operations of implants and explants.

The life cycle begins with the loading of new xenopazienti in the system. The interface developed for this feature allows a quick integration of new mice, especially if you are using the RFID reader. The cycle continues with implants, surgical procedures through which you insert a piece of tumor tissue in the body of the mouse. At this stage, you select a xenopatient and an aliquot.

An implanted mouse may be subject to measurement, with the aim of controlling the tumor growth. The managed measurements are divided in two types: qualitative and quantitative. The first is a subjective measurement, determined by the operator's discretion and obtained by palpation of xenopatient. The second is obtained electronically through a special electronic caliber, which can return the dimensions of the tumor being studied. The system minimizes the interactions between the user and the system, interfacing directly with the tools of reading. A mouse, after the measurement, can be associated with a planned treatment or for explant. The treatments allow you to test determined drugs on xenopatients, to try to determine their effectiveness. Each treatment consists of several steps, each of which corresponds to a given dose of medication. If the operator wishes to associate a set of xenopazienti to a treatment that does not exist yet, it can be defined through an interface, where a Gantt chart allows you to schedule the various steps.

The last phase of this cycle of life is represented by the explants, with the surgical procedure which removes the tumor from xenopatient, implying a sacrifice. From the removed material are created multiple fragments of tissue, which will be available for new implants on other mice. The interface implemented for simulating the real environment explants laboratory, offering a software work simple for the operator. The system provides the user the ability to interact with the screen quickly.