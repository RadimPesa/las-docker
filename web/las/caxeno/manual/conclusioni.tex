\chapter{Conclusioni}\label{chap:xenoconcl}

\Xeno\ ha come scopo principale quello di risolvere e gestire correttamente i problemi derivanti dalla sperimentazione sugli xenopazienti. \`E quindi necessario poter gestire tutte le operazioni inerenti al ciclo di vita dei topi, tra le quali sono presenti la gestione degli impianti/espianti, le misurazioni e il caricamento di nuovi individui nel sistema. Le operazioni appena descritte devono essere gestite rispettando i numerosi vincoli relazionali presenti tra i vari dati da amministrare e creare. Tutto ci\`o \`e stato risolto creando un apposito database relazionale, modellandolo in modo tale da riuscire a soddisfare al meglio le specifiche richieste dell'istituto.

L'applicazione XMS \`e stata implementata attraverso una Web Application, la quale soddisfa al meglio i precisi requisiti dell'IRCC di Candiolo. Si \`e scelta questa soluzione per poter fornire un servizio facilmente accessibile, dove gli unici requisti sono una connessione alla rete ed un Web browser. L'applicazione \`e stata sviluppata secondo il paradigma Model View Controller, mediante l'utilizzo di Django, un framework scritto in Python per la creazione di Web Application. La scelta del design pattern MVC consente di conferire al sistema una struttura modulare, permettendo cos\`i di poter intervenire in maniera puntuale e minimale in caso di modifiche dell'applicazione. Il sistema \`e stato progettato in modo tale da ottenere una distribuzione della business logic tra client e server, demandando alcune operazioni direttamente al browser dell'operatore. 

XMS offre una serie di funzionalit\`a legate allo studio e alla ricerca sugli xenopazienti. Per gestire la colonia di topi, \`e possibile inserire nuovi individui nel sistema e, eventualmente, aggiornarne lo status attuale. \`E anche possibile amministrare le operazioni di impianto, mediante le quali si immettono delle aliquote di tessuto all'interno degli xenopazienti. Inoltre, \`e necessario fornire all'utente la possibilit\`a di misurare la crescita della massa tumorale, in risposta a determinati trattamenti assegnati ai vari topi. Infine, si \`e implementata la gestione degli espianti, operazioni mediante le quali si asportono dagli xenopazienti le masse tumorali precedentemente impiantate. Questo funzionalit\`a sono utilizzabili dall'utente tramite apposite interfacce. Ognuna di queste \`e stata costruita in modo da permettere una veloce interazione con l'operatore. Inoltre, per consentire una maggiore integrazione con i sistemi in uso presso l'IRCC, si \`e gestita la compatibilit\`a con i lettori di chip RFID e di codici a barre. 

Per svolgere con maggiore efficienza i suoi compiti, \Xeno\ deve poter cooperare con gli altri moduli del LAS. In particolare, XMS accede ai dati relativi alle aliquote presenti nel sistema, creando cos\`i un flusso di comunicazione con la BioBanca. Inoltre, XMS, deve essere in grado di fornire determinate informazioni, quando richieste, alle altre applicazioni, rendendo possibili elaborazioni pi\`u dettagliate ed approfondite. Per supportare tali funzionalit\`a, \`e stata implementata una serie di Application Programming Interface (API), utili, ad esempio, per la gestione della business logic all'interno di XMS. Infatti, alcune API sono state utilizzate per effettuare controlli nel database durante l'immissione di dati da parte dell'operatore, prelevando i dati degli xenopazienti coinvolti nelle varie operazioni, ed evitando cos\`i l'inserimento di dati duplicati. Inoltre, le API risultano fondamentali per le interazioni con gli altri moduli del LAS, fornendo i dati da essi richiesti. 

Come visto nel Capitolo~\ref{chap:xenocase}, XMS offre un ambiente di lavoro in grado di simulare il laboratorio, dove ogni giorno operano i ricercatori. Questa simulazione \`e stata ottenuta soprattutto nelle interfacce relative alle operazioni di impianti ed espianti, creando schermate agevoli per gli operatori, i quali ritrovano una corrispondenza tra l'applicazione e il mondo reale.

\section{Sviluppi futuri}

Durante la progettazione e lo sviluppo di questa Web Application, si sono individuate varie possiblit\`a di ampliamento. Ogni singola miglioria contribuir\`a a perfezionare le prestazioni e i servizi offerti dal sistema.

I trattamenti sugli xenopazienti rappresentano una funzione gestita da \Xeno. Per offrire una maggior assistenza agli operatori, una possibilit\`a consiste nell'implementare un sistema di \textbf{e-mail alerting}, in modo tale da poter amministrare con maggior efficenza l'interruzione programmata dei vari trattamenti. Con questa funzione ulteriore, gli utenti (e i loro supervisori) riceverebbero una notifica nella loro casella di posta elettronica nel momento in cui sia imminente la fine di un trattamento precedentemente avviato.

Una nuova interfaccia che potrebbe essere implementata avrebbe il compito di effettaure il \textbf{check delle misure}. Si tratta di una serie di funzionalit\`a atte a verificare la bont\`a delle operazioni svolte dai vari utenti. Sar\`a accessibile solo all'amministratore del sistema, il quale potr\`a visionare tutte le azioni non ancora validate portate a termine dagli operatori. Successivamente, potr\`a scegliere se confermare, annullare o modificare le scelte prese dagli altri utenti. In questo modo, si porr\`a sulle operazioni svolte un controllo che esula dalle pure verifiche informatiche, ma che va anche ad analizzare la validit\`a scientifica dei dati.

Infine, un ultimo lavoro molto importante, sar\`a il recupero dei dati storici dell'IRCC di Candiolo, in modo tale da poter inserire nel database anche le informazioni relative agli xenopazienti degli anni passati. Questo permettar\`a a \Xeno\ di avere al suo interno un consistente storico di dati, potenziando cos\`i fin da subito la sua utilit\`a.