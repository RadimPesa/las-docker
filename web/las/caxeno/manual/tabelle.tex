\chapter{Tabelle del database}\label{chap:xenotable}

Gli attributi contrassegnati con '*' sono da considerarsi nullabili; quelli scritti in grassetto rappresentano la chiave primaria e quelli in italico sono le chiave esterne. Sono stati considerati nullabili vari attributi della tabella Mice in quanto tali dati potrebbero non essere presenti nello storico dei topi. Le tabelle contenenti la parola '\_has\_' al loro interno sono utilizzate per modellare le relazioni molti a molti tra le tabelle ad esse collegate. 

\section{Xenopazienti}

\textbf{Tabella Mice}

\begin{longtable}{|l|l|p{4.4cm}|}
\hline
\textbf{Nome campo} &	\textbf{Tipo} &	\textbf{Descrizione}\\ \hline
\textbf{id} &	BIGINT &	Identifcativo del topo\\ \hline
\textit{idMouseStrain}* &	BIGINT &	Identifcativo della razza del topo\\ \hline
\textit{idStatus} &	INT &	Identificativo dello status attuale del topo\\ \hline
barcode* &	VARCHAR(15) &	Codice del chip assegnato al topo\\ \hline
availableDate* &	DATE &	Data a partire dalla quale il topo \`e pronto per la sperimentazione\\ \hline
deathDate* &	DATE &	Data di morte del topo\\ \hline
\textit{id\_genalogy}* &	VARCHAR(15) &	Genealogia dell'impianto nel topo\\ \hline
\textit{id\_cancer\_research\_group} &	INT &	Gruppo di ricerca a cui \`e assegnato il topo.\\ \hline
\textit{id\_source}&	INT &	Chi ha fornito il topo al centro di ricerca.\\ \hline
gender &	ENUM('m','f') &	Sesso del topo.\\ \hline
birth\_date &	DATE &	Data presunta di nascita del topo.\\ \hline
notes* &	VARCHAR(150) &	Eventuali commenti (ad esempio, in caso di morte non naturale)\\ \hline
\caption{Tabella Mice}
\end{longtable}

\textbf{Tabella MouseStrain}

\begin{longtable}{|l|l|p{5.5cm}|}
\hline
\textbf{Nome campo} &	\textbf{Tipo} &	\textbf{Descrizione}\\ \hline
\textbf{id} &	INT &	Identificativo della razza del topo\\ \hline
mouse\_type\_name &	VARCHAR(50) &	Nome della razza\\ \hline
description* &	VARCHAR(150) &	Campo per eventuali descrizioni aggiuntive\\ \hline
linkToDoc* &	VARCHAR(150) & URL riferente alla documentazione ufficiale della razza\\ \hline
\caption{Tabella MouseStrain}
\end{longtable}

\textbf{Tabella Source}

\begin{longtable}{|l|l|p{5.5cm}|}
\hline
\textbf{Nome campo} &	\textbf{Tipo} &	\textbf{Descrizione}\\ \hline
\textbf{id} &	INT &	Identificativo del fornitore di xenopazienti\\ \hline
name &	VARCHAR(45) &	Nome del fornitore\\ \hline
description* &	VARCHAR(45) &	Campo per eventuali descrizioni aggiuntive\\ \hline
\caption{Tabella Source}
\end{longtable}


\textbf{Tabella Status}

\begin{longtable}{|l|l|p{6.2cm}|}
\hline
\textbf{Nome campo} &	\textbf{Tipo} &	\textbf{Descrizione}\\ \hline
\textbf{id} &	INT &	Identificativo dello stato attuale del topo\\ \hline
name &	VARCHAR(50) &	Nome dello stato attuale del topo\\ \hline
description* &	VARCHAR(150) &	Campo per eventuali descrizioni aggiuntive\\ \hline
default &	BOOLEAN &	Stato di default per i topi inseriti dalla schermata Mice Loading\\ \hline
\caption{Tabella Status}
\end{longtable}

\textbf{Tabella ChangeStatus}

\begin{longtable}{|l|l|p{5.5cm}|}
\hline
\textbf{Nome campo} &	\textbf{Tipo} &	\textbf{Descrizione}\\ \hline
\textbf{id} & INT & Chiave della tabella\\ \hline
\textit{from\_status} & INT & Status di partenza\\ \hline
\textit{to\_status} & INT & Status di arrivo\\ \hline
\caption{Tabella ChangeStatus}
\end{longtable}


\textbf{Tabella StatusInfo}

\begin{longtable}{|l|l|p{7.0cm}|}
\hline
\textbf{Nome campo} &	\textbf{Tipo} &	\textbf{Descrizione}\\ \hline
\textbf{id} &	INT &	Identificativo delle informazioni aggiuntive sullo status\\ \hline
description &	VARCHAR(45) &	Descrizione dell'informazione aggiuntiva\\ \hline
\caption{Tabella StatusInfo}
\end{longtable}

\textbf{Tabella Status\_info\_has\_status}

\begin{longtable}{|l|l|p{8.0cm}|}
\hline
\textbf{Nome campo} &	\textbf{Tipo} &	\textbf{Descrizione}\\ \hline
\textbf{id} &	INT &	Identificativo\\ \hline
id\_status &	INT &	Identificativo dello status a cui ci si riferisce\\ \hline
id\_info &	INT &	Identificativo dell'informazione aggiuntiva associata allo status.\\ \hline
\caption{Tabella Status\_info\_has\_status}
\end{longtable}

\section{Impianti - Espianti}

\textbf{Tabella Aliquots}

\begin{longtable}{|l|l|p{5.5cm}|}
\hline
\textbf{Nome campo} &	\textbf{Tipo} &	\textbf{Descrizione}\\ \hline
\textbf{id} &	BIGINT &	Identifcativo dell'aliquota\\ \hline
\textit{TypeOfTisue} &	INT &	Identficativo della tabella contenente i diversi tipi di tessuto dell'aliquota\\ \hline
\textbf{idExplant*} &	BIGINT &	Id dell'eventuale espianto da cui viene prodotta\\ \hline
id\_genealogy & VARCHAR(15) & Identificativo della genealogia a cui appartiene l'aliquota\\ \hline
\caption{Tabella Aliquots}
\end{longtable}

\textbf{Tabella TissueType}

\begin{longtable}{|l|l|p{5.5cm}|}
\hline
\textbf{Nome campo} &	\textbf{Tipo} &	\textbf{Descrizione}\\ \hline
\textbf{id} &	INT &	Identificativo del tipo di tessuto\\ \hline
\textit{abbreviation} &	VARCHAR(3) &	Abbreviazione univoca, in 3 lettere, del tipo di tessuto\\ \hline
\textit{name} &	VARCHAR(45) &	Nome completo del tipo di tessuto\\ \hline
notes* &	VARCHAR(150) &	Eventuali note\\ \hline
\caption{Tabella TissueType}
\end{longtable}

\textbf{Tabella Series}

\begin{longtable}{|l|l|p{5.5cm}|}
\hline
\textbf{Nome campo} &	\textbf{Tipo} &	\textbf{Descrizione}\\ \hline
\textbf{id} &	BIGINT &	Identificativo della serie\\ \hline
\textit{idOperator} &	BIGINT &	Identificativo dell'operatore\\ \hline
\textit{idScope} &	INT &	Identficativo della tabella contenente i vari scopi possibili\\ \hline
\textit{idTypeOfSerie} &	INT &	Identficativo della tabella contenente le varie operazioni possibili\\ \hline
date &	DATE &	Data della serie\\ \hline
notes* &	VARCHAR(150) &	Eventuali commenti\\ \hline
\caption{Tabella Series}
\end{longtable}

\textbf{Tabella Type\_of\_serie}

\begin{longtable}{|l|l|p{5.5cm}|}
\hline
\textbf{Nome campo} &	\textbf{Tipo} &	\textbf{Descrizione}\\ \hline
\textbf{id} &	INT &	Identificativo del tipo della serie\\ \hline
\textit{description} &	VARCHAR(45) &	Breve descrizione della serie \\ \hline
\caption{Tabella Type\_of\_serie}
\end{longtable}

\textbf{Tabella ImplantDetails}

\begin{longtable}{|l|l|p{7.0cm}|}
\hline
\textbf{Nome campo} &	\textbf{Tipo} &	\textbf{Descrizione}\\ \hline
\textbf{id} &	BIGINT &	Identificavitvo dell'impianto\\ \hline
\textit{idMouse} &	BIGINT &	Identificativo del topo\\ \hline
\textit{idSeries} &	BIGINT &	Identificativo della serie di impianti\\ \hline
\textit{idAliquot} &	BIGINT &	Identificativo dell'aliquota usata\\ \hline
BadQualityFlag &	BOOLEAN &	Flag per indicare la riuscita o meno dell'impianto (ad esempio, se si rovina l'aliquota mentre la si impianta)\\ \hline
site & INT & Riferimento alla tabella Site, per individuare l'impianto nel topo\\ \hline
\caption{Tabella ImplantDetails}
\end{longtable}

\subsubsection{Tabella Site}

\begin{longtable}{|l|l|p{5.5cm}|}
\hline
\textbf{Nome campo} &	\textbf{Tipo} &	\textbf{Descrizione}\\ \hline
\textbf{id} &	BIGINT &	Identificavitvo del sito dell'impianto\\ \hline
name &	VARCHAR(2) & Abbreviazione del nome del sito su due lettere\\ \hline
longName &	VARCHAR(50) & Nome del sito\\ \hline
\caption{Tabella Site}
\end{longtable}

\textbf{Tabella ExplantDetails}

\begin{longtable}{|l|l|p{5.5cm}|}
\hline
\textbf{Nome campo} &	\textbf{Tipo} &	\textbf{Descrizione}\\ \hline
\textbf{id} &	BIGINT &	Identificativo dell'espianto\\ \hline
\textit{idSeries} &	BIGINT &	Identificativo della serie di espianti\\ \hline
\textit{idMouse} &	BIGINT &	Identificativo del topo\\ \hline
\caption{Tabella ExplantsDetails}
\end{longtable}

\textbf{Tabella Programmed\_explant}

\begin{longtable}{|l|l|p{5.5cm}|}
\hline
\textbf{Nome campo} &	\textbf{Tipo} &	\textbf{Descrizione}\\ \hline
\textbf{id} &	INT &	Identificativo dell'espianto programmato\\ \hline
\textit{idScope} &	INT &	Identificativo dello scopo dell'espianto\\ \hline
\textit{idMouse} &	BIGINT &	Identificativo dello xenopaziente su cui \`e stato programmato l'espianto\\ \hline
done &	BOOLEAN &	Flag per indicare il completamento dell'espianto programmato\\ \hline
\caption{Tabella Programmed\_explant}
\end{longtable}

\textbf{Tabella ScopeDetails}

\begin{longtable}{|l|l|p{5.5cm}|}
\hline
\textbf{Nome campo} &	\textbf{Tipo} &	\textbf{Descrizione}\\ \hline
\textbf{idScope} &	INT &	Identificativo dello scopo\\ \hline
description &	VARCHAR(250) &	Descrizione dello scopo\\ \hline
\caption{Tabella ScopeDetails}
\end{longtable}

\section{Operatori}

\subsubsection{Tabella auth\_user}

\begin{longtable}{|l|l|p{6.5cm}|}
\hline
\textbf{Nome campo} &	\textbf{Tipo} &	\textbf{Descrizione}\\ \hline
\textbf{id} &	BIGINT &	Identificativo dell'operatore\\ \hline
username &	VARCHAR(30) &	Username dell'operatore\\ \hline
first\_name &	VARCHAR(20) &	Nome dell'operatore\\ \hline
last\_name &	VARCHAR(45) &	Cognome dell'operatore\\ \hline
email &	VARCHAR(30) &	Contatto dell'operatore\\ \hline
password &	VARCHAR(128) &	Password dell'operatore (salvata dopo essere stata sottoposta ad una funzione di hash)\\ \hline
is\_staff & BOOLEAN & Flag per indicare se l'utente pu\`o accedere all'interfaccia di admin\\ \hline
is\_active & BOOLEAN & Flag per indicare se l'account \`e in uso o meno; l'uso di questo flag pu\`o sostituire la cancellazione fisica dell'account nel caso in cui lo si voglia eliminare dal sistema\\ \hline
is\_superuser & BOOLEAN & Flag per indicare se l'utente \`e l'amministratore del sistema\\ \hline
last\_login & DATATIME & Data dell'ultimo login di questo operatore\\ \hline
date\_joined & DATETIME & Data in cui \`e stato creato l'account\\ \hline
\caption{Tabella auth\_user}
\end{longtable}

\subsubsection{Tabella XenoUsers }

\begin{longtable}{|l|l|p{6.5cm}|}
\hline
\textbf{Nome campo} &	\textbf{Tipo} &	\textbf{Descrizione}\\ \hline
\textit{\textbf{user\_ptr\_id}}  &	INT &	Identificativo dell'utente (operatore) a cui si riferiscono le informazioni aggiuntive.\\ \hline
\textit{id\_supervisor*} &	 INT &	Identificativo dell'eventuale supervisore\\ \hline
\textit{id\_institution} &	INT &	Identificativo dell'istituto di appartenenza.\\ \hline
\textit{id\_cancer\_research\_group} &	INT &	Identificativo del gruppo di ricerca di appartenenza.\\ \hline
\caption{Tabella XenoUsers}
\end{longtable}

\subsubsection{Tabella cancer\_research\_group}

\begin{longtable}{|l|l|p{5.5cm}|}
\hline
\textbf{Nome campo} &	\textbf{Tipo} &	\textbf{Descrizione}\\ \hline
\textbf{idCancerResearchGroup} &	INT &	Identificativo del gruppo di ricerca\\ \hline
name &	VARCHAR(255) &	Nome del gruppo di ricerca\\ \hline
\caption{Tabella cancer\_research\_group}
\end{longtable}

\subsubsection{Tabella institution}

\begin{longtable}{|l|l|p{5.5cm}|}
\hline
\textbf{Nome campo} &	\textbf{Tipo} &	\textbf{Descrizione}\\ \hline
\textbf{idInstitution} &	INT &	Identificativo dell'istituzione/ente\\ \hline
name &	VARCHAR(255) &	Nome dell'istituzione/ente\\ \hline
\caption{Tabella Institution}
\end{longtable}

\section{Gestione misurazioni}

\subsubsection{Tabella MeasurementSeries}

\begin{longtable}{|l|l|p{7.5cm}|}
\hline
\textbf{Nome campo} &	\textbf{Tipo} &	\textbf{Descrizione}\\ \hline
\textbf{id} &	BIGINT &	Identificativo della serie di misure\\ \hline
\textit{idOperator} &	BIGINT &	Identificativo dell'operatore che ha effettuato la misura\\ \hline
date &	DATE &	Data della misura\\ \hline
\textit{id\_type*} & INT & Indica con quale tipologia di misurazione sono state effettuate le misure della serie\\ \hline
\caption{Tabella MeasurementSeries}
\end{longtable}

\subsubsection{Tabella QualitativeMeasure}

\begin{longtable}{|l|l|p{6.0cm}|}
\hline
\textbf{Nome campo} &	\textbf{Tipo} &	\textbf{Descrizione}\\ \hline
\textbf{id} &	INT & Identificativo della misura qualitativa\\ \hline
\textit{id\_value} & INT &	Valore della misurazione\\ \hline
\textit{id\_series} & INT &	Serie di appartenenza della misura\\ \hline
\textit{id\_mouse} & INT &	Topo su cui \`e stata effettuata la misura\\ \hline
notes* &	VARCHAR(255) & Eventuali note sulla misurazione\\ \hline
\caption{Tabella QuantitativeMeasure}
\end{longtable}

\subsubsection{Tabella Qualitative\_values}

\begin{longtable}{|l|l|p{6.5cm}|}
\hline
\textbf{Nome campo} &	\textbf{Tipo} &	\textbf{Descrizione}\\ \hline
\textbf{id} &	INT &	Identificativo del valore qualitativo\\ \hline
value &	VARCHAR(45) &	Uno dei valori ammissibili per la misura qualitativa (es. 'Large', 'None',..)\\ \hline
description* &	VARCHAR(45) &	Eventuale descrizione.\\ \hline
\caption{Tabella Qualitative\_values}
\end{longtable}

\subsubsection{Tabella QuantitativeMeasure}

\begin{longtable}{|l|l|p{5.5cm}|}
\hline
\textbf{Nome campo} &	\textbf{Tipo} &	\textbf{Descrizione}\\ \hline
\textbf{id} &	INT &	Identificativo del valore quantitativo\\ \hline
x\_measurement &	DOUBLE &	Valore X ottenuto dalla misurazione\\ \hline
y\_measurement &	DOUBLE &	Valore Y ottenuto dalla misurazione\\ \hline
volume &	DOUBLE &	Volume ottenuto dopo un calcolo su X e Y.\\ \hline
\textit{id\_series} & INT &	Serie di appartenenza della misura\\ \hline
\textit{id\_mouse} & INT &	Topo su cui \`e stata effettuata la misura\\ \hline
notes* &	VARCHAR(255) & Eventuali note sulla misurazione\\ \hline
\caption{Tabella QuantitativeMeasure}
\end{longtable}

\subsubsection{Tabella TypeOfMeasure}

\begin{longtable}{|l|l|p{6.5cm}|}
\hline
\textbf{Nome campo} &	\textbf{Tipo} &	\textbf{Descrizione}\\ \hline
\textbf{id} &	BIGINT &	Identificativo del tipo di misura\\ \hline
name &	VARCHAR(99) &	Nome della tipologia di misura (attualmente, assume valori 'qualitative' o 'quantitative'.\\ \hline
\caption{Tabella TypeOfMeasure}
\end{longtable}

\subsection{Gestione dei trattamenti}

\subsubsection{Tabella Treatments}

\begin{longtable}{|l|l|p{5.5cm}|}
\hline
\textbf{Nome campo} & \textbf{Tipo} & \textbf{Descrizione}\\ \hline
\textbf{id} & INT & Identificativo del trattamento\\ \hline
name	 & INT & 	Nome del trattamento\\ \hline
description & INT & Breve descrizione del trattamento\\ \hline
duration	& INT & Durata prevista (in giorni o ore) del trattamento\\ \hline
type\_of\_time & ENUM(hours, days) & Questo campo serve a definire se la durata \`e specificata in ore o giorni\\ \hline
forces\_explant & BOOLEAN & Questo campo assume valore True o False a seconda del fatto che il trattamento sia acuto o meno.\\ \hline
\caption{Tabella Treatments}
\end{longtable}

\subsubsection{Tabella Detail\_treatments}

\begin{longtable}{|l|l|p{6.5cm}|}
\hline
\textbf{Nome campo} &	\textbf{Tipo} &	\textbf{Descrizione}\\ \hline
\textbf{id}	&	INT	&	Identificativo (inserito per compatibilit\`a con Django)\\ \hline
\textit{id\_via}	&	INT	&	Identifica la via di somministrazione\\ \hline
\textit{treatments\_id} &		INT	&	Identifica il trattamento di cui si stanno specificando i vari step\\ \hline
\textit{drugs\_id} &	INT	&	Indica il farmaco usato in questo step\\ \hline
start\_step	&	DATETIME	&	Data/ora di inizio di questo step\\ \hline
end\_step	&	DATETIME	&	Data/ora di fine di questo step\\ \hline
dose &	DOUBLE	&	Dose da somministrare (si suppone un numero decimale che rappresenta dei milligrammi)\\ \hline
schedule &	INT	&	Quante volte somministrare il farmarco nell'arco di tempo specificato con una riga di questa tabella\\ \hline
\caption{Tabella Detail\_treatments}
\end{longtable}

\subsubsection{Tabella Mice\_has\_treatments}

\begin{longtable}{|l|l|p{5.5cm}|}
\hline
\textbf{Nome campo} &	\textbf{Tipo} &	\textbf{Descrizione}\\ \hline
\textbf{id}	 & INT	 & Identificativo (inserito per compatibilit\`a con Django)\\ \hline
\textit{id\_mouse} &	INT & 	Topo a cui si applica il trattamento\\ \hline
\textit{id\_treatment} &  INT & Trattemento eseguito sul topo\\ \hline
\textit{id\_operator} & 	INT	 & Indica l'operatore che ha fatto partire il trattamento\\ \hline
start\_date &	DATETIME	 & Data/ora di inizio del trattamento\\ \hline
expected\_end\_date & 	DATETIME	 & Data/ora prevista per la fine del trattamento\\ \hline
end\_date* & 	DATETIME & 	Data/ora della effettiva fine del trattamento.\\ \hline
\caption{Tabella Mice\_has\_treatments}
\end{longtable}

\subsubsection{Tabella Drugs}

\begin{longtable}{|l|l|p{5.5cm}|}
\hline
\textbf{Nome campo} &	\textbf{Tipo} &	\textbf{Descrizione}\\ \hline
\textbf{id}	 & INT & 	Identificativo del farmaco\\ \hline
name & 	VARCHAR(45)	 & Nome del farmaco\\ \hline
description* & 	VARCHAR(255) & 	Eventuale descrizione del farmaco\\ \hline
\caption{Tabella Drugs}
\end{longtable}

\subsubsection{Tabella Via\_mode}

\begin{longtable}{|l|l|p{7.5cm}|}
\hline
\textbf{Nome campo} &	\textbf{Tipo} &	\textbf{Descrizione}\\ \hline
\textbf{id}	 & INT	 & Identificativo del tipo di somministrazione del farmaco\\ \hline
description* & 	INT & 	Eventaule descrizione della somministrazione del farmaco\\ \hline
\caption{Tabella Via\_mode}
\end{longtable}

\section{Supporto}

\subsubsection{Tabella Urls}

\begin{longtable}{|l|l|p{7.5cm}|}
\hline
\textbf{Nome campo} &	\textbf{Tipo} &	\textbf{Descrizione}\\ \hline
\textbf{id}	 & INT	 & Identificativo dell'URL\\ \hline
url & 	VARCHAR(255) & 	URL di destinazione\\ \hline
default & 	BOOLEAN & 	Flag per indicare quale sia l'URL attualmente selezionato\\ \hline
\caption{Tabella Urls}
\end{longtable}